
\section{For\^ets et couvertures}

Soit \textit{F} une for\^et. Soit \textit{uv} une ar\^ete de cette for\^et et le degr\'e de \textit{v} est 1.\\

Soit \textit{U} un ensemble de sommets consituant une couverture de \textit{F} de taille minimale. Si \textit{U}
couvre \textit{F}, alors \textit{U} couvre l'ar\^ete \textit{uv}, ce qui implique que \textit{u} ou bien \textit{v}
appartient \`a \textit{U}.

Nous travaillons dans une for\^et, une union disjointe d'arbres, ce qui implique que chaque sommet est soit un noeud,
soit une feuille. Chaque feuille n'a qu'un seul p\`ere, par d\'efinition, ce qui signifie qu'un sommet feuille a
obligatoirement un degr\'e \'egal \`a 1. Consid\'erons maintenant le cas inverse. Si un sommet a un degr\'e \'egal \`a
1, alors soit c'est une feuille de l'arbre, soit \c{c}a en est la racine. Dans les deux cas, un tel sommet ne peut
couvrir que lui-m\^eme et son p\`ere (ou son unique fils pour le deuxi\`eme cas). Le sommet \textit{v} consid\'er\'e
plus haut, constitue un tel sommet\\

Supposons que \textit{U} ne contient pas \textit{u}. Alors elle contient \textit{v}, et l'ar\^ete \textit{uv} est
couverte. Toutefois, le sommet \textit{u} poss\`ede probablement d'autres ar\^etes incidentes. Si \textit{F} n'est pas
limit\'ee \`a l'ar\^ete \textit{uv}, puisque \textit{v} est de degr\'e 1, alors \textit{u} poss\`ede d'autres ar\^etes
incidentes (du point de vue d'un arbre, \textit{u} poss\`ede d'autres fils et/ou a un p\`ere). Imaginons que \textit{u}
poss\`ede un autre fils feuille autre que \textit{v}, que l'on appelle \textit{w}. Alors pour que l'ar\^ete \textit{uw}
soit couverte, il faut que soit \textit{u}, soit \textit{w} appartienne \`a \textit{U}. Hors, si \texit{u} n'appartient
pas d\'ej\`a \textit{U}, alors il faut ajouter un sommet (soit \textit{u}, soit \textit{w}) \`a la couverture, ce qui
augmente sa taille.\\

Supposons maintenant que l'ar\^ete \textit{uv} n'est plus couverte par \textit{v}, mais par \textit{u}. Appelons \textit{U'}
cette couverture. Nous n'avons pas modif\'e la taille de la couverture, donc \textit{U'} est bien de taille minimale.
Reconsid\'erons maintenant cette ar\^ete \textit{uw}. Dans le cas pr\'esent, \textit{u} couvre d\'esormais cette ar\^ete.
Il existe donc bien une couverture de taille minimale qui contient \textit{u} mais pas \textit{v}.


\section{Algorithme lin\'eaire}

C'est l'algorithme auquel j'avais pens\'e (Herv\'e).
Il parcourt les sommets en profondeur une seule fois. Le seul b\'emol est qu'il faut au pr\'ealable avoir pars\'e le gengraph.
Si \`c{c}a fait partie de l'algo, ben c'est un peu mort (\`a moins de l'adapter).


\begin{verbatim}
couvArbre (r : entier, liste_couv : liste_d_entiers) : entier {

  booleen couvrant := faux ;
  
  pour chaque fils de r {
    entier s := fils(r);
    si nombre_de_fils (s) = 0  // si s est une feuille
      couvrant := vrai;        // r doit faire partie de la couverture
    sinon
      si (non couvArbre(s), liste_couv)    //Si s n'est pas une feuille, on regarde s'il fait partie de la couverture.
        couvrant := vrai;                  //Si ce n'est pas le cas, alors r doit couvrir l'arete rs, puisque s ne le fait pas
  }
  
  si (couvrant := vrai)
    ajouter(list_couv, r);
  retourner couvrant;  
}
\end{verbatim}

