
\section{Algorithme 2-approch\'e sur des graphes}

Soit \textit{G} un graphe connexe non orient\'e. Soit \textit{T} l'arbre orient\'e obtenu par une recherche en profondeur sur \textit{G}.
Si \textit{G} est connexe, alors \textit{$\forall u,v \in V(G), \exists chemin de u a v$}. Cela implique que tous les sommets
feront partie de l'arbre \textit{T}.\\

Soit \textit{T} l'arbre orient\'e correspondant \`a une arborescence possible de \textit{G}. Tout sommet est soit une feuille,
soit un sommet. Or, dans un arbre, les relations de p\`ere \`a fils (c'est-\`a-dire les ar\^etes entre deux sommets) sont soient
des relations de type noeud/noeud, ou bien noeud/feuille (repr\'ecisons que la notion de noeud \'equivaut \`a un sommet non feuille).
Donc, si l'on retourne l'ensemble des noeuds de l'arbre, comme couverture de cet arbre, alors toutes les relations dans l'arbre (soient
toutes les ar\^etes) sont couvertes. Cet algorithme retourne bien une couverture du graphe \textit{G}.


\subsection{Algorithme bas\'e sur le couplage maximum}

Le but de cet algorithme est de calculer un ensemble d'ar\^etes (que l'on appelle \textit{M} tel qu'aucune d'entre elles n'ait une
extr\'emit\'e en commun. L'autre particularit\'e est qu'on ne peut pas ajouter d'ar\^etes \`a cet ensemble sans que la premi\`ere
propri\'et\'e \'enonc\'ee juste avant ne devienne fausse. Ainsi, si on ne peut pas rajouter d'ar\^etes \`a cet ensemble, cela siginifie
que pour chaque ar\^ete n'appartenant pas \`a \textit{M}, l'une des deux extr\'emit\'es est incluse dans \textit{M}. Soit \textit{U} les
extr\'emit\'es des ar\^etes appartenant \`a \textit{M}. Si une ar\^ete appartient \`a \textit{M}, alors elle est couverte par ses
deux extr\'emit\'es. Si elle n'y appartient pas, alors elle est couverte par l'une de ses deux extr\'emit\'es. Donc, \textit{U}
est bien \'equivalent \`a une couverture de sommets.

Notre algorithme de couplage, consultable dans le fichier source, se contente de parcourir la liste des ar\^etes du graphe. Pour une ar\^ete
donn\'ee, si aucune de ses extr\'emit\'es n'appartient \`a la couverture actuelle, alors on les ins\`ere dedans. Si l'une des deux extr\'emit\'es
y appartient d\'ej\`a, alors on n'ins\`ere rien du tout. La complexit\'e en temps de calcul est tributaire du nombre total d'ar\^etes.

\subsection{Algorithme bas\'e sur la recherche en profondeur}

Nous avons trouv\'e deux algorithmes qui donnent le m\^eme r\'esultat
mais de diff\'erentes mani\`eres. Nous pr\'esentons l'algorithme qui suit
puisque l'autre est consultable depuis le fichier source.

\begin{verbatim}
parcoursProfondeur(int r, vector<list<int>> & list_succ, vector<bool> & visited):vide
  visited[r] := vrai;
  pour toute arete incidente a r faire
    s := voisin[r];
    si (visited[s] = faux) alors
      ajouter(list_succ[r], s);
      parcoursProfondeur(s, list_succ, visited);
    finsi
  finpour
fin


couvProfondeur(vector<list<int>> & list_succ):liste_de sommets(reference)
  liste_couv : liste_de_sommets
  pour chaque sommet r de list_succ
    si (list_succ[r] != vide)
      ajouter(liste_couv, r);
    finsi
  finpour
  retourner liste_couv;
fin
\end{verbatim}

Pour ce qui est de la complexit\'e, la proc\'edure de parcours en profondeur va \^etre en \textit{$O(2*E(G))$} (\textit{E(G)} \'etant le nombre
d'ar\^etes du graphe \textit{G}). En effet, pour un sommet \textit{s}, on regarde chaque voisin. Si ce voisin (disons \textit{v}) n'a pas \'et\'e
visit\'e, on explore celui-ci avant d'en finir avec \textit{s}. Comme on examine chaque voisin de chaque sommet, on va explorer les
voisins de \textit{v}, et on va donc retomber sur \textit{s}. Comme \textit{s} a \'et\'e marqu\'e comme visit\'e, on ne va pas le r\'eexplorer,
mais cette suite d'op\'erations \'equivaut \`a visiter deux fois l'ar\^ete \textit{sv}. En appliquant ce raisonnement \`a toutes les ar\^etes,
on obtient donc une complexit\'e d'ordre \textit{$O(2*E(G))$}. Hors, dans le cas le plus co\^uteux o\^u \textit{G} est un graphe parfait,
il va contenir \textit{$n-1$} ar\^etes \`a visiter pour le premier sommet, puis \textit{$n-2$} pour le deuxi\`eme, et ainsi de suite jusqu'au
dernier sommet. Le nombre d'ar\^etes correspond alors au calcul de la suite des \textit{$n-1$} premiers entiers. On peut alors \'ecrire :
\textit{$ E(G) = ( n*(n-1)) / 2 $}. On obtient donc une complexit\'e de \textit{$O(n*(n-1))$} (puisqu'on parcourt l'\'equivalent du double
du nombre d'ar\^etes).\\

La fonction \textit{couvProfondeur} parcourt le vecteur de sommets (de taille le nombre de sommets, soit\textit{n}), et se contente de mettre
dans une liste les sommets non-feuille, qui feront partie de la couverture. Cette fonction a donc une complexit\'e de \textit{$O(n)$}.\\

La complexit\'e de l'algorithme est donc de l'ordre de \textit{$O( n*(n-1) + n )$}, soit \textit{$O(n^2)$}.\\

Pour l'algorithme que nous avons impl\'ement\'e, la complexit\'e est
\'egalement en O($n^{2}$).