% -*- mode: latex; coding: latin-1-unix -*- %

\section{Algorithme r\'ecursif couverture taille minimale}

Le principe de cet algorithme est assez simple. Nous avons une liste
de feuilles, une liste contenant les sommets appartenant � la
couverture, un tableau contenant les p\`eres des sommets, un tableau
contenant le nombre de fils des sommets, et enfin la racine de
l'arbre.

Cet algorithme est de complexit\'e lin\'eaire. Cet algorithme n'a pas
besoin de savoir quel sommet est voisin d'un autre, il faut juste
savoir qui est p\`ere de qui. Cet algorithme se finit quand la seule feuille
restante est la racine, cet algo va d\'etruire l'arbre au fur et �
mesure de son d\'eroulement.

On commence par prendre une feuille, on cherche son grand p\`ere. Un
sommet a toujours un p\`ere, sauf la racine, donc on se passe de la
v\'�rification du p\`ere. On coupe l'ar\^ete entre i et son p\`ere, et
entre le p\`ere et le grand p\`ere et on ajoute le p\`ere � la
couverture. Enfin si k est devenu une feuille on le rajoute \'a la lise
de feuilles.

\begin{verbatim}
liste feuilles;
liste couverture;
int[] pere;
int[] nbFils;
int racine;

function void algo(){
// on part du principe que la liste ne peut pas �tre vide.
int i := feuilles.premier();
Si i := racine;
  alors fin algo;

int j := pere[i];
//il n'est pas possible qu'un sommet autre que la racine n'ait pas de
p�re.
int k := pere[j];
//on retire i de la liste de feuilles.
feuilles.supprimerEnTete();
// i est une feuille, on ajoute son p�re dans la couverture.
couverture.ajouterEnTete(j);
//On "coupe" l'arr�te entre i et son p�re, j.
pere[j] := 0;
//On "coupe" l'arr�te entre j et son p�re, k.
pere[i] := 0;

Si k != 0
   alors nbFils[k]--;
         Si nbFils[k] = 0
            alors feuilles.ajouterEnTete(k);

algo();

}
\end{verbatim}

