% -*- mode: latex; coding: latin-1-unix -*- %

\section{Prototype}

Voici l'algo r�cursif auquel j'ai pens�. La syntaxe n'est pas correcte.

Voil� notre algorithme r�cursif :

\bigskip
map <key, iterator> graph;

int[] couverture;

int nb\_sommet = 0;

\bigskip
function booleen algo1(var noeud, nb\_fils:entiers){\\
  booleen couvert = vrai;\\
  Pour chaque fils du sommet \og noeud \fg faire\\
     couvert *= algo1(nom\_du\_fils,degre\_du\_fils);\\
  FindePour\\
  Si couvert == faux faire\\
     couvert = vrai;\\
     couverture[nb\_sommet]=noeud;\\
     nb\_sommet++;\\
  FindeSi\\
  Si degre\_du\_fils == O;\\
    couvert = faux;\\
  FindeSi\\
  retourner ((noeud.nb\_de\_fils >= nb\_fils) + couvert);\\
}

\bigskip
Exemple de fonctionnement de notre m�thode sur notre graphe exemple.

Prenons cet arbre :\\
0-1-3\\
|\\
2-4-7\\
|\\
5\\
|\\
6

\newpage
Dans notre variable graph on aura donc :

[0] -> [ 1 - 2 ]

[1] -> [ 3 ]

[2] -> [ 4 - 5]

[3] -> null

[4] -> [7]

[5] -> [6]

[6] -> null

Voici les appels r�cursifs a notre m�thode algo1 :

algo1(0,0); // noeud 0 et il a 2 fils

premier passage dans la boucle Pour :

algo1(1,2); // noeud 1 et le noeud 0 a 2 fils

> algo1(3,1); // noeud 3 et la noeud 1 a 1 fils

>\ > on ne passe pas dans la boucle Pour (3 n'a pas de ils)

>\ > Comme 3 n'a pas de fils on met couvert � \og faux \fg

>\ > On renvoie \og faux \fg au processus appelant

> retour au processus appelant // noeud 1

> on sort de notre Pour

> Couvert est maintenant � \og faux \fg

> on met couvert � vrai

> on ajoute le noeur 1 � la couverture de sommet

> on incr�mente nb\_sommet

> on renvoie vrai au processus appelant

retour au processus appelant // noeud 0

couvert reste � \og vrai \fg

deuxi�me tour de boucle

algo1(2,2) // noeud 2 et le noeud 0 � 2 fils

> Premier tour de boucle

> algo1(4,2) // noeud 4 et le noeud 2 � deux fils

>\ > Premier tour de boucle

>\ > algo1(7,1) // noeud 7 et le noeud 4 a un fils

>\ >\ > on ne passe pas dans la boucle Pour (7 n'a pas de fils)

>\ >\ > Comme 7 n'a pas de fils on met couvert � \og faux \fg

>\ >\ > On renvoie \og faux \fg au processus appelant

>\ > retour au processus appelant // noeud 4

>\ > couvert est maintenant � \og faux \fg

>\ > on sort de notre boucle (car plus de fils)

>\ > on met couvert � \og vrai \fg

>\ > on ajoute le noeud 4 a notre couverture de sommet

>\ > on incr�mente nb\_sommet

>\ > on renvoie \og vrai \fg au processus appelant

> retour au processus appelant // noeud 2

> couvert reste \og vrai \fg

> Deuxieme tour de boucle

> algo1(5,2) // noeud 5 et le noeud 2 a 2 fils

>\ > Premier tour de boucle

>\ > algo1(6,1) // noeud 6 et le noeud 5 a 1 fils

>\ >\ > on ne passe pas dans la boucle Pour (6 n'a pas de fils)

>\ >\ > Comme 6 n'a pas de fils on met couvert a faux

>\ >\ > on renvoie \og faux \fg au processus appelant

>\ > retour au processus appelant // noeud 5

>\ > couvert est maintenant a \og faux \fg

>\ > on sort de notre boucle (5 n'a plus de fils)

>\ > on met couvert � \og vrai \fg

>\ > on incr�mente nb\_sommet

>\ > on ajoute le noeud 5 � notre couverture de sommet

>\ > on incr�mente nb\_sommet

>\ > on renvoie \og vrai \fg au processus appelant

> retour au processus appelant // noeud 2

> couvert reste a \og vrai \fg

> on a parcouru tous les noeuds de 2 on sort de la boucle Pour

> on renvoie \og vrai \fg au processus appelant

retour au processus appelant // noeud 0

couvert reste � \og vrai \fg

On a parcouru tous les fils du noeud 0 on sort de la boucle Pour.

Notre algo est termin� et la couverture de sommet se trouve dans notre variable \og couverture \fg
