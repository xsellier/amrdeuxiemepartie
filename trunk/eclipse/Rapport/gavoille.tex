% -*- mode: latex; coding: latin-1-unix -*- %

\section{Le super algo de fou avec les tableaux}

\begin{verbatim}

int size;

liste feuilles;
liste couverture;
int[] pere;
int[] nbFils;
int racine;

function void algo(){
// on part du principe que la liste ne peut pas �tre vide.
int i := feuilles.premier();
Si i := racine;
  alors fin algo;

int j := pere[i];
//il n'est pas possible qu'un sommet autre que la racine n'ait pas de
p�re.
int k := pere[j];
//on retire i de la liste de feuilles.
feuilles.supprimerEnTete();
// i est une feuille, on ajoute son p�re dans la couverture.
couverture.ajouterEnTete(j);
//On "coupe" l'arr�te entre i et son p�re, j.
pere[j] := 0;
//On "coupe" l'arr�te entre j et son p�re, k.
pere[i] := 0;

Si k != 0
   alors nbFils[k]--;
         Si nbFils[k] = 0
            alors feuilles.ajouterEnTete(k);

algo();

}
 

\end{verbatim}

