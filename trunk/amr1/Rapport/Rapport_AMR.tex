\documentclass{article}

\usepackage[latin1]{inputenc}
\usepackage[francais]{babel}
\usepackage{url}
\usepackage{fullpage}
\usepackage{fancyhdr}
\usepackage{cite}
\usepackage{graphics}
\usepackage{graphicx}
\usepackage{color}
\pagestyle{fancy}
\headheight=12pt
\lhead{Algorithmes du monde r\'eel}
\chead{}
\rhead{Plugins C++}
\renewcommand{\headrulewidth}{0.5pt}
\lfoot{\today}
\cfoot{}
\rfoot{Master 2 Informatique semestre 1 - AMR}
\renewcommand{\footrulewidth}{0.5pt}
\selectfont

\hoffset -0.21in
\voffset -0.724in
\textheight 23cm
\headheight 0.8cm
\headsep 1cm
\topmargin 0in
\textwidth 17.cm
\parskip 10pt

\title{Algorithmes du monde r\'eel\\Reductions de probl\`emes NP-Complets}
\author{Descombe Herv\'e \and Desenne Nicolas \and Sellier Xavier}
\begin{document}
\maketitle
\newpage
\tableofcontents
\newpage

\section{Chemin et circuit Hamiltoniens}

Pour r\'eduire ces probl\`emes, nous avons fait le choix de suivre les conseils indiqu\'es sur
la feuille de TD et de consid\'erer des variables repr\'esentant la position i d'un sommet j
dans un chemin (circuit) Hamiltonien. Nous allons consid\'erer tout d'abord le probl\`eme du
chemin Hamiltonien, puis dans un second temps, nous nous pencherons sur celui du circuit,
qui diff\`ere tr\`es peu au niveau de la r\'eduction.

\subsection{Chemin Hamiltonien}

\subsubsection{R\'eduction vers SAT}

Dans un graphe, un chemin est Hamiltonien si et seulement s'il passe une seule fois par tous les sommets
de ce graphe. Ainsi, soit \textit{n} le nombre de sommets d'un graphe \textit{G}. Si ce chemin passe une
seule fois par tous les sommets, il va donc \^etre consitu\'e de \textit{n} sommets ; il y aura donc
\textit{n} positions possibles. On consid\`ere $x_i,j$ la variable bool\'enne indiquant si le sommet
\textit{j} est situ\'e \`a la position \textit{i} d'un tel chemin. Sur cette base, on peut d\'efinir un
ensemble r\`egles permettant de sp\'ecifier \`a partir de variables bool\'eennes si un chemin est
Hamiltonien :

%1)
%\[
%	 \forall j\in V(G)\left( $x_{1j}$ \vee $x_{2j}$ \vee \ldots \vee $x_{nj}$ \right)
%\]

Cette premi\`ere r\`egle indique que chaque sommet est associ\'e \`a au moins une position dans le chemin.
C'est une condition n\'ecessaire puisque le chemin Hamiltonien est cens\'e parcourir tous les sommets.

2)% \forall i avec 1 <= k <= n, \big( $x_i,1$ \vee $x_i,2$ \vee ... \vee $x_i,n$ ) ;

A l'inverse de la premi\`ere r\`egle, celle-ci indique que toute position dans un chemin Hamiltonien doit
\^etre associ\'ee \`a au moins un sommet.

3) %\forall i avec 1 <= i <= n, \forall j, k \in V(G) avec j \ne k, \big( \neg $x_i,j$ \vee \neg $x_i,k$ ) ;

Cette r\`egle sp\'ecifie que deux sommets ne peuvent pas se trouver \`a la m\^eme position dans un chemin
Hamiltonien.

4)% \forall i avec 1 <= i < n, \exists j,k \in V(G) avec j \ne k et {j,k} \in E(G), \big( $x_i,j$ \wedge $x_i+1,k$ ) ;

Cette r\`egle d\'etermine que pour toute position \textit{i} d'un chemin Hamiltonien, il existe un sommet adjacent \`a
celui courant qui se trouve \`a la position \textit{i+1} du chemin. C'est cette derni\`ere r\`egle qui permet
d\'eliminer beaucoup de cas dans lesquels on ne peut pas atteindre un sommet sans passer par ceux d\'ej\`a visit\'es
(qui implique qu'il sont \`a une position inf\'erieure \`a la position courante). On peut remarquer que l'on n'a pas
parmi ces quatre r\`egles de restriction explicite sur le fait qu'un m\^eme sommet peut se trouver \`a deux positions
diff\'erentes. En fait, les r\`egles 1), 2) et 4) impliquent cette restriction : on a \textit{n} sommets et \textit{n}
positions ; si on attribue deux positions diff\'erentes \`a un m\^eme sommet, on va se retrouver avec un chemin \`a
\textit{n} positions mais \textit{n-1} sommets. Or, la r\`egle 4) pr\'ecise que pour chaque position \textit{k} de 1 \`a
\textit{n}, il existe un successeur diff\'erent. Pour que cette condition soit respect\'ee, il faut donc qu'il y ait
\textit{n} sommets distincts dans le chemin Hamiltonien. Le fait de ne pas expliciter cette r\`egle va en plus
\^etre utile pour le cas du circuit Hamiltonien.\\

Pour en revenir \`a cette quatri\`eme r\`egle, cette formulation ne convient pas pour une traduction triviale vers le
format \textit{minisat} (qui prend des conjonctions de disjonctions). On va donc modifier la formule afin de pouvoir
la traduire sans trop de difficult\'es en disjonction de variables bool\'eennes :

4 bis)% \forall i avec 1 <= i <= n, \forall j,k \in V(G) avec {j,k} \notin E(G), \big( \net $x_i,j$ \vee \net $x_i+1,k$ ) ;

Maintenant, on sp\'ecifie que si deux sommets ne sont pas adjacents, ils ne peuvent pas se trouver \`a deux positions
cons\'ecutives dans le chemin.\\

\subsubsection{Algorithme du programme}

La fonction \textit{hamiltonian\_path} applique les r\`egles explicit\'ees ci-dessus sur notre matrice de bool\'eens
repr\'esentant les ar\^etes entre les sommets. Il n'y a en fait que la quatri\`eme r\`egle qui explore la matrice,
puisqu'il faut v\'erifier s'il existe une ar\^ete entre deux sommets. Nous pr\'ef\'erons ici expliquer la mani\`ere
dont nous avons cod\'e les variables bool\'eennes pour les inclure dans le fichier destin\'e \`a \^etre utilis\'e
par \textit{minisat}.\\

Comme il y a \textit{n} positions et \textit{n} sommets, on va avoir \textit{$n \times n$} configurations possibles, soit
\textit{$n \times n$} variables bool\'eennes dans notre fichier de sortie. On avait plusieurs possibilit\'es de coder
un sommet \textit{j} \`a une position \textit{i}. Nous avons choisi la mani\`ere suivante :
\begin{itemize}
 \item  les \textit{n} premiers entiers codent les \textit{n} positions pour le sommet 1 ; par exemple, si l'entier \textit{2}
est positif (soit que la variable qu'il repr\'esente est vraie), alors le sommet 1 se trouve \`a la deuxi\`eme position dans
le chemin ; si l'entier est n\'egatif, alors il ne se trouve pas \`a cette position ;
 \item  les entiers de \textit{n+1} \`a \textit{2n} codent les positions pour le sommet 2 ;
 \item ...
 \item pour un sommet k, ses positions sont repr\'esent\'ees par les entiers situ\'es dans l'intervalle [
\textit{(k-1)*n+1} ; \textit{k*n}]. 
\end{itemize}

Typiquement, pour la premi\`ere r\`egle, on aura une premi\`ere ligne dans le fichier r\'esultat qui va signifier
que le sommet 1 peut prendre au moins une des \textit{n} positions dans le chemin (pour un graphe de taille \textit{4} :
\begin{verbatim}
1 2 3 4 0
\end{verbatim}

L'algorithme de la fonction est d\'ecoup\'e en quatre parties distinctes correpondant \`a l'application des quatres r\`egles.
Nous ne trouvons pas utile d'expliciter les autres d\'etails de l'impl\'ementation.

\subsection{Tests et r\'esultats}

Nous avons pu tester la validit\'e de nos r\`egles et de notre algorithme sur des graphes de petites tailles.
% Voir pour ajouter des jeux de tests a executer directement.

Nous avons fait des tests de charge. Pour un graphe de 100 sommets, le programme met moins d'une seconde \`a traduire le fichier
d'entr\'ee en un nouveau fichier au format SAT. Pour un graphe de 1000 sommets, par contre, il n\'ecessite <A REMPLIR> secondes
d\'ex\'ecution.

\subsection{Circuit Hamiltonien}

Le probl\`eme du circuit Hamiltonien est tr\`es proche de son pr\'ed\'ecesseur. Dans ce cas, on ne cherche plus un chemin, mais
un cycle qui passe une seule fois par chaque sommet du graphe. Toutefois, on peut consid\'erer ce probl\`eme comme un cas
particulier du chemin Hamiltonien \`a ceci pr\`es que les sommets situ\'es en premi\`ere et derni\`ere positions doivent \^etre
adjacents. Ainsi, il suffit de rajouter cette r\`egle \`a l'ensemble d\'ej\`a d\'efinis. \textit{n} correspond toujours au
nombre de sommets dans le graphe, soit au nombre de positions possibles.

5)% \exists k,j \in V(G), {k,j} \in E(G) et \big( $x_1,j$ \wedge $x_n,k$ ) ;

Encore une fois, cette formulation ne s'adapte pas \`a la transformation vers le format SAT. On reformule la r\`egle comme
suit :

5 bis) %\forall j,k \in V(G) avec {j,k} \notin E(G), \big(\net$x_1,j$ \vee \net $x_n,k$ ) ;

On remarque cependant que cette r\`egle est un cas particulier de la r\`egle 4) du chemin Hamiltonien.\\

Dans l'algorithme utilis\'e pour la r\'eduction, cela se traduit par un petit ajout \`a effectuer dans le traitement de
la quatri\`eme r\`egle. La fonction \textit{hamiltonian\_circuit} code cet algorithme. C'est la m\^eme que celle
du chemin Hamiltonien exception faite de l'ajout \`a effectuer, qui se traduit par deux lignes \`a rajouter dans le
fichier SAT \`a chaque fois que deux sommets ne sont pas adjacents.

Au niveau des tests,  on a l\`a-aussi test\'e sur des graphes de petites tailles, et en particulier sur des graphes qui
poss\'edaient un plusieurs chemins Hamiltoniens mais pas de circuit.
Nous n'avons pas jug\'e utile de proc\'eder \`a des tests de charge puisque la diff\'erence avec le probl\`eme pr\'ec\'edent
provient de quelques op\'erations suppl\'ementaires s'ex\'ecutant en temps constant.

\end{document}