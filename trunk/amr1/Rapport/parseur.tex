Pour charger un graph nous avons implémenté un analyseur syntaxique
qui va lire le fichier passé en premier argument du programme, le
découper en arrêtes, ces arrêtes sont insérées dans un vecteur
temporaire avant d'être stockés dans une matrice.

La syntaxe à utiliser pour entrer un graph est assez simple. On peut
soit entrer des listes d'arrêtes séparées par un espace ou un retour à
la ligne. On peut aussi entrer des chemins séparés aussi par un espace
ou un retour à la ligne. Mais surtout pas d'espace en fin de ligne.

Voici un exemple de fichier de graph:\\
0-1-2-3-4-5-6-7-8-9\\
9-10-11-12-13 13-14\\

Notre programme va lui même lancer minisat, analyser le résultat et
afficher un résultat à l'écran. Il créé deux fichiers. Le fichier
minisat.txt est le fichier contenant la réduction, c'est le fichier
d'entrée de minisat. Le deuxième fichier, result.txt est le fichier de
sortie de minisat. Le programme n'efface pas ces fichiers car, si
l'utilisateur le souhaite, il est possible de rajouter l'inverse de la
solution trouvée dans minisat.txt et de relancer minisat pour voir si
il n'y a pas d'autres solutions.
