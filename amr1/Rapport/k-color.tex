Le problème de la k-coloriabilité, permet de déterminer si un graphe
peut être colorié avec k couleurs sans que deux sommets voisins aient
la même couleur.

Pour réduire ce problème à SAT, il faut autant de variables que de
sommets multiplié par le nombre de couleurs. 

Il faut que pour chaque sommet i et j tels que i et j soient voisins,
Ai -> \neg Aj, Ai signifiant que le sommet i a la couleur A. L'implication
peut être remplacé par un ou ce qui donne \neg Ai ou \neg Aj. 

Il faut aussi qu'un sommet n'ai qu'une seule couleur donc pour chaque
sommet i, Ai -> \neg Bi et  Ai -> \neg Ci ... Comme pour l'implication
précédente pour être testée on remplace l'implication par \neg Ai ou Bi.

Il faut aussi qu'un sommet ai une couleur donc on a pour chaque sommet
i Ai ou Bi ou Ci ...
